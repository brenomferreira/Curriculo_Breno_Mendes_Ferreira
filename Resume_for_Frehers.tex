%%%%%%%%%%%%%%%%%%%%%%%%%%%%%%%%%%%%%%%%%
% Medium Length Professional CV
% LaTeX Template
% Version 2.0 (8/5/13)
%
% This template has been downloaded from:
% http://www.LaTeXTemplates.com
%
% Original author:
% Rishi Shah 
%
% Important note:
% This template requires the resume.cls file to be in the same directory as the
% .tex file. The resume.cls file provides the resume style used for structuring the
% document.
%
%%%%%%%%%%%%%%%%%%%%%%%%%%%%%%%%%%%%%%%%%

%----------------------------------------------------------------------------------------
%	PACKAGES AND OTHER DOCUMENT CONFIGURATIONS
%------------------------------------------------------------------------


\documentclass{resume} % Use the custom resume.cls style

\usepackage[left=0.75in,top=0.6in,right=0.75in,bottom=0.6in]{geometry} % 

\usepackage[utf8]{inputenc}
\usepackage[brazilian]{babel}

\usepackage{etoolbox}
\patchcmd{\thebibliography}{\section*{\refname}}{}{}{}

\usepackage{hyperref}
    \hypersetup{
        colorlinks=true,
        linkcolor=blue,
        filecolor=magenta,      
        urlcolor=cyan,
    }

%Document margins
\newcommand{\tab}[1]{\hspace{.2667\textwidth}\rlap{#1}}
\newcommand{\itab}[1]{\hspace{0em}\rlap{#1}}
\name{Breno Mendes Ferreira} % Your name
\address{Engenheiro Eletricista - Campo Grande, MS, Brasil} % Your address
%\address{123 Pleasant Lane \\ City, State 12345} % Your secondary addess (optional)
\address{(61)98381-4472 \\ brenomferreira@gmail.com \\ \the\year\  }% Your phone number and email
\address{CV LATTES: \url{http://lattes.cnpq.br/9876193060151800}
}





\begin{document}


%----------------------------------------------------------------------------------------
%	EDUCATION SECTION
%----------------------------------------------------------------------------------------

\begin{rSection}{Resumo}

\textbf{Doutorando} do Programa de Pós-Graduação em Engenharia de Sistemas Eletrônicos e de Automação do Departamento de Engenharia Elétrica da Universidade de Brasília (UnB) na área de \textbf{Engenharia Biomédica} (Reabilitação com Eletroestimulação): 
Desenvolve pesquisa junto ao Laboratório de Automação e Robótica (LARA) e ao grupo Empoderando Mobilidade & Autonomia (EMA).
\textbf{Mestre} em Engenharia Elétrica pelo Programa de Pós-Graduação em Sistemas de Energia pela Universidade Tecnológica Federal do Paraná (UTFPR-Curitiba) na área de Instrumentação Eletrônica (Ultrassom / FPGA):
Desenvolve pesquisa em ultrassom junto ao Laboratório de Ultrassom (UTFPR-Curitiba) desde 2015; Entre as áreas de pesquisa destaca-se: Instrumentação eletrônica aplicada ao ultrassom; caracterização de materiais utilizando ultrassom; caracterização de transdutores de ultrassom.
\textbf{Graduação}: Formou em Engenharia Elétrica na UTFPR-Curitiba: Realizou Iniciação Científica na área de Sistemas de Comunicações Sem Fio; Ainda na graduação realizou o programa de Trainee Inova Talentos (CNPq) na empresa Robert Bosch GmbH, Curitiba/PR.
\\ Areas de Atuação Acadêmica: Engenharia Biomédica; Instrumentação Eletrônica.

\end{rSection}
\begin{rSection}{Educacional}

\\{\bf Universidade de Brasília, Brasília/DF} \hfill {\em 2017 - Atual} 
\\ \small{Programa de Pós-Graduação em Engenharia de Sistemas Eletrônicos e de Automação (PGEA)}
\\ \textbf{Doutorando em Engenharia Biomédica}

\\{\bf Universidade Tecnológica Federal do Paraná, Curitiba/PR} \hfill {\em 2017} 
\\ \small{Departamento de Engenharia Elétrica - Programa de Pós-Graduação em Sistemas de Energia (PPGSE)}
\\ \textbf{Mestrado em Engenharia Elétrica}

{\bf Universidade Tecnológica Federal do Paraná, Curitiba/PR} \hfill {\em 2015} 
\\ Departamento de Engenharia Elétrica
\\\textbf{Engenharia Elétrica}


% Minor in Linguistics \smallskip \\
% Member of Eta Kappa Nu \\
% Member of Upsilon Pi Epsilon \\


\end{rSection}

% \begin{rSection}{Carrier Objective}
%  To work for an organization which provides me the opportunity to improve my skills and knowledge to grow along with the organization objective.
% \end{rSection}
%--------------------------------------------------------------------------------
%    Projects And Seminars
%-----------------------------------------------------------------------------------------------
\begin{rSection}{Participação em Projetos} 

{\bf Pesquisador e Desenvolvedor de projetos no grupo EMA} \hfill {\em 2017 - Atual}
\\\textit{Vínculo: Colaborador; Nível: Doutorando; Cidade: Brasília/DF.}
\\Desenvolve tecnologias para aperfeiçoar a reabilitação de pessoas com deficiências motoras. Em especial, trabalhos focados no uso da estimulação elétrica para pessoas com lesões e doenças que afetam o sistema nervoso.(www.ene.unb.br/antonio/ema/pt/)

{\bf Pesquisador no Grupo de Pesquisa LARA} \hfill {\em 2017 - Atual}
\\\textit{Vínculo: Colaborador; Nível: Doutorando; Cidade: Brasília/DF.}
\\Desenvolve pesquisa e divulgação científica em workshops desenvolvidos pelo programa de pós-graduação do departamento de Engenharia Elétrica UnB. (http://lara.unb.br/)

{\bf Pesquisador CNPq} \hfill {\em 2015 - 2017}
\\\textit{Vínculo: Colaborador; Nível: Mestrando; Carga horária: 20; Regime: Parcial; Cidade: Curitiba/PR.}
\\Realizou pesquisa em Ultrassom junto ao Programa de Pós-graduação em Sistema de Energia (PPGSE) da Universidade Tecnológica Federal do Paraná (UTFPR). Entre as áreas de pesquisa, destacam-se: instrumentação eletrônica aplicada ao ultrassom; caracterização de materiais utilizando ultrassom e caracterização de transdutores de ultrassom.\\

{\bf Iniciação Científica} \hfill {\em 2014 (6 meses)}
\\\textit{Vínculo: Voluntário; Nível: Graduando; Carga horária: 10 h/sem; UTFPR; Orientador: Guilherme Luiz Moritz; Cidade: Curitiba/PR.}
\\Realizou atividades no Departamento CPGEI, na área de Sistemas de Comunicações Sem Fio (ZIGBEE) no laboratório LABSC, com atribuições em pesquisa, elaboração de \textit{hardware}, desenvolvimento de \textit{software} ambos na área de Redes de Sensores sem Fio.






\end{rSection}
%----------------------------------------------------------------------------------------
%	TECHNICAL STRENGTHS SECTION
%----------------------------------------------------------------------------------------



%----------------------------------------------------------------------------------------
%	WORK EXPERIENCE SECTION
%----------------------------------------------------------------------------------------

\begin{rSection}{Atuação Profissional}



{\bf Docência - UnB} \hfill {\em 2017/2018 (total, 12 meses)}

\textit{Vínculo: Bolsista; Enquadramento funcional: Estagiário; Carga horária: 4h/sem; Regime: Parcial; Orientador: Antônio Padilha L. Bó; Cidade: Brasília/DF.}
\\Atuou como professor em duas turmas da disciplina Laboratório de Dispositivos e Circuitos Eletrônicos durante os semestres 2017-1 e 2018-1, totalizando 120 horas de aula. Sua atuação como professor na disciplina se deu sob a supervisão direta e no contexto de atividade obrigatória no Programa de Pós-Graduação de Engenharia de Sistemas Eletrônicos e de Automação (PGEA/UnB), onde atuou com o seu orientador.

{\bf Engenheiro Trainee - VZP Projetos e Execuções} \hfill {\em 2015 (9 meses)}

\textit{Vínculo: Contratado; Enquadramento funcional: Engenheiro Eletricista; Carga horária: 20h/sem; Cidade: Curitiba/PR}
\\Realizou orçamentos, planejamento, monitoramento de execuções de projetos elétricos e de Automação Residencial / Industrial. Liderando equipe de eletricista técnicos nas execuções de obras elétricas.


{\bf Trainee CNPq -  Robert Bosch GmbH} \hfill {\em 2014 - 2015}

\textit{Vínculo: Bolsista; Regime: Parcial; Programa de Trainee Inova Talentos IEL/CNPq; Projeto: Bancada Didática de Motores Automotivos; Tutor: Eduardo Kamaroski Neto; Cidade: Curitiba/PR.}
\\Participou do desenvolvimento e planejamento do projeto de um produto protótipo. Com treinamentos específico de motores a Diesel, programação em unidades de controle eletrônico e Estudos de desenvolvimento de Aplicativos Mobile. Além de cursos oferecidos para o desenvolvimento do lado comportamental pessoal. Responsabilidades: Programação, parte estrutural elétrica e conteúdo didático empregado no produto. 



\end{rSection}


%	EXAMPLE SECTION
%----------------------------------------------------------------------------------------

\begin{rSection}{Idiomas}

\textbf{Inglês:} Compreende Bem; Fala Razoavelmente; Escreve Razoavelmente; Lê Bem. 

\end{rSection}

\begin{rSection}{Habilidades Específicas}

% \begin{tabular}{ @{} >{\bfseries}l @{\hspace{6ex}} l }
% Modeling and Analysis  & AutoCad, Revit, StaadPro \\
% Software \& Tools & MS Office, Latex \\
% \end{tabular}

\begin{tabular}{ @{} >{\bfseries}l @{\hspace{6ex}} l }
Modelagem/Programação: & MatLab; AutoCad; Android Studio; Fusion360; Eagle \\
Softwares: & Office; Latex \\
Linguagens de Desenvolvimento: &  Python; C++; Matlab; Java \\
\end{tabular}

\end{rSection}

\newpage

\begin{rSection}{Publicações Acadêmicas} 

\nocite{DaFonseca2019}

\nocite{DaFonseca2019a}

\nocite{Assef2019}

\nocite{Assef2018}

\nocite{Assef2016}

\nocite{assefmodelagem}



\bibliographystyle{IEEEtran}

\bibliography{references}


\end{rSection}


%----------------------------------------------------------------------------------------
% Extra Curricular
%----------------------------------------------------------------------------------------


\end{document}
